\chapter{Struttura della Lega}
\label{cap:struttura-lega}

\section{Finalità e principi ispiratori}
\label{art:1.1}

La Lega FantaScarrupat è un’associazione amatoriale a carattere ludico, ispirata ai valori dell’amicizia, del rispetto reciproco e della partecipazione condivisa. La Lega si fonda su principi democratici: ogni squadra partecipa con pari dignità alla vita della competizione, contribuendo attivamente alle decisioni comuni.\\
La gestione del gioco è affidata alla piattaforma digitale “Fantacalcio.it”.

\section{Composizione e ruolo del Consiglio di Lega}
\label{art:1.2}

Tutti i fantallenatori iscritti alla Lega fanno parte del Consiglio di Lega, organo deliberativo collegiale incaricato di adottare le decisioni strategiche, organizzative e regolamentari. Il Consiglio stabilisce le regole della stagione, convoca e gestisce le aste, e risolve eventuali controversie.\\

Per la stagione sportiva 2025/2026, il Consiglio risulta composto da otto membri. La Presidenza è affidata ad Antonio Di Giorgio, in qualità di amministratore della Lega. Ciascun membro del Consiglio dispone di pari diritti di voto e doveri di partecipazione.

\section{Il Presidente della Lega}
\label{art:1.3}

Il Presidente è il garante del corretto svolgimento delle attività della Lega. Oltre a convocare e presiedere le riunioni, egli ha facoltà di:
\begin{itemize}
  \item proporre votazioni straordinarie o d’emergenza in caso di necessità;
  \item segnalare e verbalizzare comportamenti contrari allo spirito sportivo della Lega;
  \item escludere temporaneamente un membro dalle votazioni, qualora quest’ultimo assuma atteggiamenti offensivi, ostruzionistici o in aperta violazione del regolamento, previa comunicazione motivata al Consiglio.
\end{itemize}
\noindent
L'esclusione temporanea dovrà essere ratificata da una votazione straordinaria entro 72 ore.

\chapter{Votazioni di Lega}
\label{cap:Votazioni}

\section{Votazioni ordinarie}
\label{art:2.1}

Le votazioni ordinarie regolano le modifiche al presente regolamento e tutte le decisioni non urgenti che influiscano sulla gestione generale della Lega. Esse possono svolgersi in presenza oppure tramite canali digitali (es. messaggistica), a condizione che sia garantita la tracciabilità dei voti.\\

Per essere valida, la votazione deve prevedere la partecipazione di tutti i membri del Consiglio. La proposta è approvata se ottiene almeno la maggioranza assoluta (50\% + 1 dei voti favorevoli). In caso di parità o mancata approvazione, resta in vigore la regola precedente.

\section{Votazioni straordinarie}
\label{art:2.2}

Le votazioni straordinarie sono convocate quando si presenta un'urgenza dovuta a fattori estranei alla condotta dei partecipanti (per esempio slittamenti del calendario ufficiale, problemi tecnici della piattaforma, indisponibilità logistica della sede d'asta) che richiede di adattare rapidamente l'organizzazione senza compromettere la regolarità del gioco.\\

La richiesta può provenire direttamente dal Presidente o tramite un membro del consiglio; la delibera è approvata se riceve il voto favorevole di due terzi dei membri aventi diritto.\\

Il Presidente conserva in ogni momento il potere di sospendere o escludere temporaneamente dal voto qualunque consigliere che tenga comportamenti lesivi, linguaggio offensivo o ostruzionismo deliberato.

\section{Votazioni d’emergenza}
\label{art:2.3}

Le votazioni d’emergenza si attivano quando un fatto riconducibile al comportamento di uno o più membri (quale abbandono immotivato, reiterata mancata consegna della formazione, accordi illeciti o altra violazione grave) pone a rischio la regolarità, la trasparenza o la continuità della stagione.\\

La convocazione spetta esclusivamente al Presidente, che ne illustra tempestivamente le motivazioni. La proposta si considera approvata con il voto favorevole di tre quarti dei membri, oppure di due terzi qualora i restanti risultino formalmente irreperibili o assenti da almeno quarantotto ore.\\

Anche in questa procedura il Presidente può sospendere o escludere temporaneamente dalla deliberazione il consigliere che, con parole o atti, ostacoli il corretto svolgimento della votazione.


\chapter{Sistema Sanzionatorio}
\label{cap:sistema-sanzionatorio}

\section{Principi generali}
\label{art:3.1}

Il presente capitolo disciplina il regime sanzionatorio applicabile ai partecipanti della Lega FantaScarrupat in caso di comportamenti contrari al regolamento, ai principi di correttezza sportiva o allo spirito della competizione.\\
Ogni sanzione deve essere proporzionata alla gravità dell’infrazione e applicata nel rispetto della trasparenza, della parità di trattamento e del diritto di difesa.

\section{Tipologie di infrazioni}
\label{art:3.2}

\noindent
Costituiscono infrazioni sanzionabili, a titolo esemplificativo e non esaustivo:
\begin{itemize}
  \item la mancata consegna reiterata della formazione senza giustificato motivo;
  \item comportamenti o linguaggi offensivi nei confronti degli altri partecipanti o del Presidente;
  \item la simulazione o il tentativo di manipolare l’esito del campionato;
  \item l’ostruzionismo deliberato nelle votazioni o nei processi decisionali;
  \item la diffusione non autorizzata di dati personali o comunicazioni riservate della Lega;
  \item il mancato rispetto degli obblighi economici connessi alla partecipazione alla Lega.
\end{itemize}

\section{Misure sanzionatorie}
\label{art:3.3}

\noindent
In funzione della gravità e della recidiva, possono essere applicate le seguenti sanzioni:
\begin{itemize}
  \item \textbf{Richiamo ufficiale}: ammonimento formale.
  \item \textbf{Penalizzazione in crediti}: riduzione del budget per l’asta di riparazione o la stagione successiva.
  \item \textbf{Penalizzazione in classifica}: sottrazione di punti.
  \item \textbf{Sospensione temporanea}: esclusione da una o più giornate, con formazione automatica.
  \item \textbf{Aumento della quota annuale}: maggiorazione fino al 50\% della quota ordinaria per recidiva o gravi inadempienze.
  \item \textbf{Esclusione definitiva}: rimozione del partecipante con eventuale annullamento dei risultati.
\end{itemize}

In caso di esclusione definitiva, qualora la quota stagionale non risulti già versata e non vi siano modalità efficaci di riscossione, l’importo dovuto sarà equamente ripartito tra tutti gli altri membri della Lega.

\section{Procedura di applicazione}
\label{art:3.4}

Le sanzioni sono deliberate dal Consiglio mediante votazione straordinaria, con quorum di almeno i due terzi dei votanti favorevoli.\\
L’interessato ha diritto a essere preventivamente informato e a presentare le proprie difese entro un termine definito (24–48 ore). In caso di silenzio, si procederà d’ufficio.\\
Tutte le sanzioni devono essere comunicate ufficialmente e regolarmente verbalizzate.

\section{Ricorsi e riesame}
\label{art:3.5}

Il partecipante sanzionato può chiedere un riesame entro 72 ore dalla notifica della sanzione. Il riesame sarà discusso e votato con gli stessi criteri dell’applicazione iniziale. In mancanza di richiesta, la sanzione diventa definitiva.


\chapter{Il Gioco}
\label{cap:il-gioco}

\section{Modalità di gioco}
\label{art:4.1}

Il campionato si svolge secondo la modalità Classic del regolamento ufficiale di Fantacalcio, applicato attraverso la piattaforma “Fantacalcio.it”. Le regole di calcolo dei punteggi, le valutazioni e le funzionalità tecniche seguono quanto previsto dal sito di riferimento, fatte salve le modifiche e integrazioni previste dal presente regolamento.

\section{Composizione della rosa}
\label{art:4.2}

\noindent
Ogni squadra deve costituire una rosa di 25 calciatori, così suddivisi:

\begin{itemize}
  \item 3 portieri
  \item 8 difensori
  \item 8 centrocampisti
  \item 6 attaccanti
\end{itemize}

\noindent
La composizione avviene durante l’asta iniziale e deve rispettare tale suddivisione in ogni momento della stagione.

\section{Asta iniziale}
\label{art:4.3}

\subsection{Prima fase: Regole e votazioni}
\label{art:4.3.1}

Prima dell’avvio dell’asta vera e propria, il Consiglio di Lega si riunisce per discutere eventuali modifiche regolamentari e votare il metodo di svolgimento dell’asta, valido sia per l’asta iniziale che per le aste successive.

\subsection{Seconda fase: Composizione delle rose}
\label{art:4.3.2}

Ogni squadra dispone di un budget iniziale di 1.000 crediti per la formazione della propria rosa.\\
L’asta si svolge per ruoli in ordine: portieri, difensori, centrocampisti, attaccanti.\\
Il metodo di chiamata dei giocatori può avvenire secondo una delle seguenti modalità, da scegliere tramite votazione durante la prima fase:
\begin{itemize}
  \item Scorrimento alfabetico dei cognomi;
  \item Chiamata casuale tramite software “FantaAsta”;
  \item Chiamata diretta libera da parte dei fantallenatori, secondo l’ordine stabilito.
\end{itemize}

\noindent
Ogni giocatore parte da una base d’asta di 1 credito.

\subsection{Obbligo di completamento per ruolo}
\label{art:4.3.3}

Prima di passare da un reparto all’altro, tutti i partecipanti devono aver completato il numero esatto di calciatori previsti per quel ruolo.\\
Ad esempio, non sarà possibile iniziare l’asta dei difensori finché ogni squadra non abbia acquistato esattamente 3 portieri. Tale regola si applica anche agli altri reparti, nel rispetto della suddivisione stabilita.

\subsection{Chiamata diretta integrativa}
\label{art:4.3.4}

Al termine dello scorrimento dell’elenco per ciascun ruolo, le squadre che non abbiano completato il reparto possono proseguire con una fase integrativa di chiamata diretta, selezionando liberamente i giocatori mancanti secondo l’ordine concordato.

\subsection{Correzione errori e svincoli}
\label{art:4.3.5}

Al termine dell’asta di ciascun reparto, è possibile effettuare correzioni:

\begin{itemize}
  \item I fantallenatori possono svincolare giocatori non desiderati e recuperare interamente il credito speso.
  \item Lo svincolo deve essere dichiarato immediatamente e in modo irrevocabile.
  \item I giocatori svincolati possono essere riacquistati solo tramite nuova asta, in cui possono partecipare anche altri fantallenatori.
  \item Se nessun altro partecipa alla nuova asta, il giocatore può essere riacquistato al prezzo originariamente pagato.
\end{itemize}

\noindent
La lista ufficiale dei ruoli è quella pubblicata da Fantacalcio.it.

\subsection{Errori di assegnazione}
\label{art:4.3.6}

In caso di errore nell’assegnazione di un calciatore (giocatore già appartenente a un’altra rosa, oppure non presente nel listone ufficiale), l’aggiudicazione è da considerarsi nulla.\\
Il credito speso sarà integralmente restituito e il giocatore sarà nuovamente disponibile all’acquisto, secondo le modalità previste.

\section{Schieramento della formazione}
\label{art:4.4}

La formazione va schierata entro 15 minuti dall’orario di inizio della prima partita del turno di Serie A.\\

Ciascuna formazione titolare deve comprendere 11 calciatori, secondo uno dei seguenti moduli ammessi:

\begin{itemize}
  \item 3-4-3
  \item 3-5-2
  \item 4-3-3
  \item 4-4-2
  \item 4-5-1
  \item 5-4-1
  \item 5-3-2
\end{itemize}

\noindent
La panchina è composta da 11 giocatori, così distribuiti:

\begin{itemize}
  \item 2 portieri
  \item 3 difensori
  \item 3 centrocampisti
  \item 3 attaccanti
\end{itemize}

In caso di mancata consegna della formazione, il sistema recupera automaticamente l’ultima formazione valida disponibile.

\section{Sostituzioni}
\label{art:4.5}

Il sistema gestisce automaticamente le sostituzioni in base all’ordine della panchina e alla compatibilità di ruolo.\\
Se un titolare non ottiene voto oppure è indicato come “s.v.”, viene sostituito dal primo giocatore disponibile dello stesso ruolo in panchina.\\
In assenza di sostituti disponibili, viene assegnato un voto d’ufficio pari a 0 (zero).\\
Sono consentite fino a cinque sostituzioni per ogni giornata di campionato.

\section{Recidiva nella mancata consegna della formazione}
\label{art:4.6}

Qualora un partecipante ometta la consegna della formazione per più di tre giornate complessive, anche non consecutive, durante una stessa stagione, il Presidente potrà avviare una procedura sanzionatoria ai sensi del Capitolo 3 del presente regolamento.



\chapter{Mercato}
\label{cap:mercato}

\section{Prima asta di riparazione (opzionale)}
\label{art:5.1}

Qualora l’asta iniziale si svolga prima della chiusura ufficiale del calciomercato estivo, verrà organizzata una prima asta di riparazione, al termine della sessione di mercato, con l’obiettivo di aggiornare le rose in base ai trasferimenti dell’ultima fase. \\

\noindent \textbf{Crediti disponibili:}

Ogni squadra partecipa con il residuo di crediti dell’asta iniziale, sommato ai crediti ottenuti dallo svincolo dei calciatori dichiarati prima della sessione.\\

\noindent \textbf{Modalità di svolgimento:}

L’asta si svolge per reparto (portieri, difensori, centrocampisti, attaccanti) e può essere condotta secondo uno dei seguenti metodi, scelti tramite votazione nella prima asta stagionale:
\begin{itemize}
  \item Scorrimento alfabetico dei cognomi;
  \item Chiamata casuale tramite software “FantaAsta”;
  \item Chiamata diretta libera dei fantallenatori.
\end{itemize}

\noindent \textbf{Regole per gli svincoli e i riacquisti:}
\begin{itemize}
  \item Prima dell’inizio dell’asta, ciascun fantallenatore deve comunicare, per l’intera rosa, i giocatori che intende svincolare; tali giocatori saranno immediatamente considerati liberi e i relativi crediti restituiti integralmente.
  \item Un giocatore svincolato potrà essere riacquistato solo attraverso nuova asta.
  \item In mancanza di rilanci da parte di altri fantallenatori, il giocatore potrà essere riacquistato al prezzo originario.
\end{itemize}

\noindent  \textbf{Obbligo di sistemazione delle rose:}

Prima dell’asta, ogni squadra deve risultare in regola con i criteri previsti dal sistema “Le Priorità” (Art. 5.4), se applicabili.

\section{Seconda asta di riparazione (obbligatoria)}
\label{art:5.2}

La seconda asta di riparazione si tiene al termine del calciomercato invernale, ed è obbligatoria per tutte le squadre. Le regole sono identiche alla prima asta (Art. 5.1), salvo diversa deliberazione del Consiglio.

\noindent
Anche in questo caso:
\begin{itemize}
  \item Si utilizzano i crediti residui della stagione, inclusi quelli derivanti da svincoli;
  \item È necessario comunicare in anticipo l’intera lista dei calciatori da svincolare, senza limitazioni al solo reparto oggetto dell’asta;
  \item Si applicano le modalità d’asta definite inizialmente;
  \item Le rose devono essere regolarizzate secondo le Priorità (Art. 5.4) e Sessioni di scambio (Art. 5.3) prima dell'inizio.
\end{itemize}

\section{Sessioni di scambi}
\label{art:5.3}

Durante fasi specifiche della stagione, il Presidente può indire sessioni ufficiali di scambio tra squadre, previa comunicazione a tutti i partecipanti.

\noindent \textbf{Fasi abilitate agli scambi:}
\begin{itemize}
  \item Periodi di sosta per le nazionali;
  \item Periodi compresi tra un’asta e la giornata successiva.
\end{itemize}

\noindent \textbf{Regole generali:}
\begin{itemize}
  \item Gli scambi devono avvenire tra giocatori dello stesso ruolo;
  \item Il valore d’acquisto segue il calciatore trasferito;
  \item È ammesso includere una somma di crediti come conguaglio tra le parti, purché venga dichiarata nel dettaglio.
\end{itemize}

\noindent
Ogni tentativo di elusione o simulazione può comportare votazione d’emergenza e sanzioni.\\

\noindent \textbf{Procedura obbligatoria:}
\begin{itemize}
  \item Comunicazione ufficiale su WhatsApp da parte di uno dei due fantallenatori, indicando:
  \begin{itemize}
    \item giocatori coinvolti;
    \item crediti originari d’acquisto;
    \item eventuale conguaglio in crediti.
  \end{itemize}
  \item Conferma dell’altro fantallenatore nella stessa conversazione.
  \item Verifica e ufficializzazione a cura del Presidente, con messaggio pubblico nella chat Telegram e aggiornamento dei crediti.
  \item Lo scambio sarà valido solo dopo l’ufficializzazione da parte dell’amministrazione.
\end{itemize}

\noindent \textbf{Limiti tecnici:}\\
Nel caso in cui uno scambio comporti il superamento del limite massimo di 1.000 crediti totali, l’admin adatterà il valore dei giocatori acquisiti per riportare il budget entro i limiti previsti.

\section{Le priorità}
\label{art:5.4}

Il sistema delle “Priorità” è applicabile in caso di uscita di un calciatore dalla Serie A. Il fantallenatore che ne detiene i diritti può usufruire di una sostituzione agevolata. \\

\noindent \textbf{Condizioni per l’attivazione della priorità:}
\begin{itemize}
  \item Il calciatore deve risultare ufficialmente trasferito fuori dalla Serie A;
  \item La sostituzione è ammessa con un calciatore svincolato, di pari ruolo, con quotazione uguale o inferiore a quella del giocatore uscito, calcolata al momento della richiesta di sostituzione;
  \item Il giocatore scelto deve essere presente nella lista svincolati dalla data dell’ultima asta ufficiale svolta.
\end{itemize}

\noindent \textbf{Modalità:}

Il nuovo calciatore verrà acquistato al medesimo prezzo del giocatore sostituito; l’operazione deve essere comunicata e approvata dal Presidente, che ne verifica i criteri.



\chapter{Gestione Infortuni e Indisponibilità}
\label{cap:gestione-infortuni}

\section{Definizioni generali}
\label{art:6.1}

\noindent
Ai fini del presente regolamento:

\begin{itemize}
  \item Si definisce \textbf{infortunato} un calciatore che, a causa di una condizione fisica o mentale accertata, non risulta convocabile per un numero significativo di partite consecutive.
  \item Si definisce \textbf{indisponibile} un calciatore escluso dalle convocazioni per ragioni non mediche, tra cui motivi tattici, disciplinari, societari o di altra natura (es. fuori rosa, sospeso, squalificato).
\end{itemize}

\noindent
La gestione dei casi dipende dalla durata della non convocazione e dalla sua causa.

\section{Infortuni di durata inferiore ai 4 mesi}
\label{art:6.2}

Se un calciatore in rosa subisce un infortunio la cui durata effettiva (compresa tra la data del primo stop e quella della prima nuova convocazione) è inferiore a 4 mesi, il fantallenatore non potrà sostituire il giocatore né ricorrere ad alcun meccanismo di supporto.

\section{Infortuni di durata pari o superiore ai 4 mesi}
\label{art:6.3}

Nel caso in cui la durata dell’infortunio sia pari o superiore a 4 mesi, oppure venga ufficialmente dichiarata la fine anticipata della stagione per il calciatore in questione, il fantallenatore ha diritto a sostituirlo tramite la lista degli svincolati.

\noindent \textbf{Condizioni per la sostituzione:}
\begin{itemize}
  \item È sufficiente che una diagnosi ufficiale, pubblicata da fonti attendibili (es. società calcistica, Fantacalcio.it, principali testate sportive), indichi un’assenza prevista pari o superiore a 4 mesi;
  \item Una volta approvata la sostituzione, non sarà più possibile annullarla, anche in caso di rientro anticipato del calciatore;
  \item Il calciatore sostitutivo deve essere di pari ruolo, presente nella lista svincolati dalla data dell’ultima asta ufficiale;
  \item La quotazione di riferimento del giocatore da sostituire è quella in vigore il giorno dell’infortunio;
  \item Il sostituto deve avere una quotazione pari o inferiore a tale valore.
\end{itemize}

\noindent  \textbf{Limite temporale:}

Il diritto alla sostituzione può essere esercitato fino al 31 marzo compreso. Dal 1° aprile in poi, nessuna sostituzione sarà più ammessa, anche in presenza di infortuni gravi o diagnosi di stagione conclusa.

\section{Calciatori indisponibili}
\label{art:6.4}

I calciatori considerati indisponibili per ragioni non mediche (es. esclusione per scelta tecnica, fuori rosa, squalifica prolungata, sospensione contrattuale) non possono essere sostituiti.\\
Il fantallenatore dovrà attendere il reintegro del calciatore o provvedere durante una regolare finestra di mercato.

\section{Casi di positività al SARS-CoV-2}
\label{art:6.5}

I calciatori risultati positivi al virus SARS-CoV-2 (COVID-19) sono gestiti come infortunati ordinari, senza differenze regolamentari.\\
Pertanto, si applicano tutte le disposizioni del presente Capitolo 7, in funzione della durata dell’indisponibilità.



\chapter{Interruzione del Campionato di Serie A}
\label{cap:interruzione-campionato}

\section{Sospensione straordinaria del campionato}
\label{art:7.1}

In caso di sospensione imprevista o straordinaria del campionato di Serie A (es. emergenze sanitarie, eventi eccezionali, motivazioni istituzionali), la Lega FantaScarrupat attenderà l’eventuale ripresa ufficiale delle partite per proseguire regolarmente la stagione.\\

Se la ripresa avviene con modalità differenti da quelle originarie (es. campionato compresso, tornei a eliminazione, playoff/playout), il gioco si interromperà all’ultima giornata completa disputata secondo il format iniziale.\\

In caso di ripresa posticipata, tutti i dati regolamentari (punteggi, penalità, scambi, sostituzioni, modificatori, ecc.) restano validi, salvo diversa decisione del Consiglio di Lega.

\section{Validità delle competizioni e gestione delle quote}
\label{art:7.2}

\noindent
\textbf{a) Se la stagione raggiunge almeno il 60\% delle giornate previste:}
\begin{itemize}
  \item La stagione è considerata valida;
  \item I premi previsti verranno ricalcolati proporzionalmente in base al numero effettivo di giornate giocate;
  \item Le quote eccedenti (cioè non utilizzate per premi e spese) saranno restituite oppure compensate sulla quota della stagione successiva.
\end{itemize}

\noindent
\textbf{b) Se la stagione non raggiunge il 60\% delle giornate previste:}
\begin{itemize}
  \item La stagione è considerata annullata ai fini sportivi;
  \item I premi vengono completamente annullati e non saranno distribuiti;
  \item Le quote versate saranno integralmente restituite oppure compensate sulla stagione successiva;
  \item L’albo d’oro sarà aggiornato regolarmente, ad eccezione delle competizioni con finale unica non disputata, che non verranno conteggiate.
\end{itemize}

\section{Sospensioni programmate e pause da calendario}
\label{art:7.3}

Le presenti disposizioni non si applicano nei casi in cui il campionato venga interrotto per motivi già previsti dal calendario ufficiale, come:
\begin{itemize}
  \item Mondiali o Europei per nazionali;
  \item Competizioni continentali (es. Coppa d’Africa, Copa América);
  \item Olimpiadi o altri eventi internazionali ufficiali.
\end{itemize}

In questi casi, il gioco si sospende automaticamente in attesa della regolare ripresa del campionato, senza modifiche al regolamento o al calendario delle competizioni.

\section{Votazione Straordinaria}
\label{art:7.4}

In presenza di interruzioni impreviste, modifiche al format della Serie A o contesti non disciplinati da questo articolo, il Presidente ha facoltà di convocare una Votazione Straordinaria (vedi Art. 2.2) per deliberare su:
\begin{itemize}
  \item Eventuale chiusura anticipata della stagione in corso;
  \item Ricalcolo dei premi e delle quote in modalità alternativa;
  \item Possibile inserimento o modifica di competizioni extra.
\end{itemize}


\chapter{Competizioni}
\label{cap:competizioni}

\section{Serie A}
\label{art:8.1}

Alla competizione “Serie A” partecipano tutte le 8 squadre della Lega.\\
Il torneo si svolge con girone asimmetrico e calendario generato in modo casuale, dal presidente di lega in presenza di almeno un altro membro del consiglio come testimone, tramite l’apposito software di Fantacalcio.it, durante la fase d’Asta Iniziale.

\noindent \textbf{Punteggio:}
\begin{itemize}
  \item 3 punti per la vittoria;
  \item 1 punto per il pareggio;
  \item 0 punti per la sconfitta.
\end{itemize}

\noindent \textbf{Classifica finale e premi:}\\
Le prime 3 classificate sono considerate vincitrici e accedono alla distribuzione dei premi stagionali.

\noindent
In caso di parità in classifica, si applicano i seguenti criteri nell’ordine:
\begin{itemize}
  \item Somma punti totali;
  \item Classifica avulsa;
  \item Differenza reti;
  \item Gol fatti;
  \item Gol subiti.
\end{itemize}

\section{Champions League}
\label{art:8.2}
\noindent Partecipano tutte le 8 squadre, suddivise in due fasi:

\subsection{Fase a gironi}
\label{art:8.2.1}

Le squadre vengono divise in 2 gruppi da 4, generato in modo casuale insieme al calendario, dal presidente di lega in presenza di almeno un altro membro del consiglio come testimone, tramite l’apposito software di Fantacalcio.it, durante la fase d’Asta Iniziale.\\
Ogni squadra disputa 6 incontri, affrontando le avversarie del proprio gruppo.

\noindent \textbf{Punteggio gironi:}
\begin{itemize}
  \item 3 punti per la vittoria;
  \item 1 punto per il pareggio;
  \item 0 punti per la sconfitta.
\end{itemize}

\noindent
\noindent \textbf{Criteri in caso di parità nei gironi:}
\begin{itemize}
  \item Somma punti;
  \item Classifica avulsa;
  \item Differenza reti;
  \item Gol fatti;
  \item Gol subiti.
\end{itemize}

\noindent
Le prime due classificate di ciascun girone accedono alla fase successiva.

\subsection{Eliminazione diretta}
\label{art:8.2.2}

Le prime classificate sfidano le seconde del girone opposto con gare di andata e ritorno. La finale si gioca in partita secca.

\noindent \textbf{Criteri per determinare la vincente:}
\begin{itemize}
  \item Somma dei punteggi delle due gare;
  \item Supplementari (vedi Art. 8.4);
  \item Rigori (vedi Art. 8.4);
  \item Somma punti totali di campionato;
  \item Gol in trasferta.
\end{itemize}

\section{NBA}
\label{art:8.3}

\noindent Questa competizione coinvolge tutte le 8 squadre ed è suddivisa in:

\subsection{Regular Season}
\label{art:8.3.1}

Si disputa con girone asimmetrico di 14 giornate, con calendario generato in modo casuale, dal presidente di lega in presenza di almeno un altro membro del consiglio come testimone, tramite l’apposito software di Fantacalcio.it, durante la fase d’Asta Iniziale.

\noindent \textbf{Punteggio:}
\begin{itemize}
  \item 3 punti per la vittoria;
  \item 1 punto per il pareggio;
  \item 0 punti per la sconfitta.
\end{itemize}

\noindent \textbf{Criteri di parità in classifica:}
\begin{itemize}
  \item Somma punti totali;
  \item Classifica avulsa;
  \item Differenza reti;
  \item Gol fatti;
  \item Gol subiti.
\end{itemize}

\noindent
Tutte le 8 squadre si qualificano alla fase successiva.

\subsection{Play-Off}
\label{art:8.3.2}

Competizione a eliminazione diretta, con andata e ritorno fino alla semifinale e finale secca.

\noindent \textbf{Tabellone (basato sulla Regular Season):}
\begin{itemize}
  \item Match 1: 1ª vs 8ª
  \item Match 2: 4ª vs 5ª
  \item Match 3: 3ª vs 6ª
  \item Match 4: 2ª vs 7ª
\end{itemize}

\noindent
\textbf{Semifinali:}
\begin{itemize}
  \item Match 5: vincente Match 1 vs vincente Match 2
  \item Match 6: vincente Match 3 vs vincente Match 4
\end{itemize}
\textbf{Finale:} vincente delle due semifinali

\noindent \textbf{Criteri di vittoria:}
\begin{itemize}
  \item Risultato totale tra andata e ritorno;
  \item Supplementari (vedi Art. 8.4);
  \item Rigori (vedi Art. 8.4);
  \item Somma punti totali;
  \item Gol in trasferta.
\end{itemize}

\section{Supplementari e Rigori}
\label{art:8.4}

Le regole sui supplementari e i rigori si applicano esclusivamente alle fasi a eliminazione diretta delle competizioni in cui sono espressamente previste, ovvero Champions League (fase KO) e NBA (Play-Off).\\
Non sono previsti per le gare della fase a gironi, della Regular Season o per eventuali recuperi o spareggi non ufficiali.

\subsection{Supplementari}
\label{art:8.4.1}

Ogni squadra calcola un punteggio supplementare basato sulla media delle fantamedie complessive (comprensive di bonus e malus) dei migliori quattro panchinari (escluso il portiere) non subentrati in campo.\\
Se meno di quattro giocatori ottengono voto valido, si integrano con voti d’ufficio pari a 5,5 fino a raggiungere il numero richiesto.

\begin{longtable}{|c|c|}
  \hline
  \textbf{Media complessiva} & \textbf{Gol supplementari} \\
  \hline
  0.00 -- 6.49 & 0 gol \\ \hline
  6.50 -- 6.99 & 1 gol \\ \hline
  7.00 -- 7.49 & 2 gol \\ \hline
  +0.50 ulteriore & +1 gol aggiuntivo \\
  \hline

  \caption{Gol x Media Supplementari}
\end{longtable}

\subsection{Rigori}
\label{art:8.4.2}

Se i supplementari non determinano una vincitrice, si procede con i rigori.\\
Il risultato di ciascun rigore è calcolato esclusivamente in base al voto puro (senza bonus o malus) del calciatore.

\begin{itemize}
  \item Il calciatore segna se il voto puro è maggiore o uguale a 6;
  \item Sbaglia se il voto è inferiore a 6.
\end{itemize}

Ogni squadra esegue 5 rigori. In caso di ulteriore parità, si procede ad oltranza, un rigore alla volta per squadra, fino a che una delle due prevale.

\noindent \textbf{Ordine di esecuzione dei rigori:}
\begin{itemize}
  \item Attaccanti titolari con voto valido (in ordine di schieramento);
  \item Centrocampisti titolari con voto valido;
  \item Difensori titolari con voto valido;
  \item Portiere titolare con voto valido;
  \item Riserve subentrate con voto valido (in ordine panchina).
\end{itemize}


\chapter{Quote, Penalità e Premi}
\label{cap:quote-penalita-premi}

\section{Quote di partecipazione}
\label{art:9.1}

Per prendere parte alla stagione, ogni squadra corrisponde una quota complessiva di 150 euro, versata al Tesoriere della Lega.\\
L’importo è frazionato in due rate uguali da 75 euro: la prima va pagata entro il giorno in cui si tiene la Seconda Asta di Riparazione, mentre la seconda è dovuta in occasione dell’ultima giornata di Serie A, quando i partecipanti sono soliti ritrovarsi per il consueto aperitivo di fine campionato, offerto dalle squadre vincitrici.\\
Al momento di ciascun pagamento il Tesoriere rilascia una semplice ricevuta—non fiscale ma nominativa—che costituisce l’unica prova ufficiale dell’avvenuto versamento.

\section{Penalità per ritardato pagamento}
\label{art:9.2}

La mancata corresponsione di una rata entro la relativa scadenza comporta, a carico della squadra inadempiente, una penalizzazione in classifica pari al numero di giornate di Serie A già disputate alla data di scadenza non rispettata.

\section{Premi}
\label{art:9.3}

Le quote annuali (8 squadre × € 150 = € 1 200) costituiscono il montepremi da distribuire come segue:

\begin{longtable}{|l|l|r|}
  \hline
  \textbf{Competizione} & \textbf{Posizione / Titolo} & \textbf{Premio (€)} \\
  \hline
  Serie A & 1ª & 350 \\
  & 2ª & 270 \\
  & 3ª & 200 \\ \hline
  Champions League & Vincitrice & 190 \\ \hline
  NBA & Vincitrice & 190 \\
  \hline
  \caption{Premi}
\end{longtable}

\section{Adeguamenti in caso di stagione abbreviata}
\label{art:9.4}

\noindent
Le disposizioni dell’Art. 8.2 si applicano anche al montepremi:

\begin{itemize}
  \item \textbf{Stagione valida} (almeno il 60\% delle giornate giocate): i premi sono proporzionati alle giornate disputate; eventuali quote eccedenti sono restituite o compensate sulla quota della stagione successiva.
  \item \textbf{Stagione annullata} (meno del 60\% delle giornate giocate): tutti i premi sono annullati; le quote versate sono restituite o compensate sulla stagione seguente.
\end{itemize}

\section{Fondo di riserva}
\label{art:9.5}

Eventuali residui non distribuiti (arrotondamenti, penalità pecuniarie rientrate, ecc.) confluiscono in un fondo di riserva gestito dal Tesoriere.\\
L’impiego del fondo richiede votazione ordinaria e può essere destinato a:
\begin{itemize}
  \item spese straordinarie di Lega;
  \item incremento dei premi della stagione successiva;
  \item altro.
\end{itemize}



\chapter{Sito Web e Identità delle Fantasquadre}
\label{cap:sito-identita}

\section{Portale ufficiale}
\label{art:10.1}

Dal campionato 2022-2023 la Lega FantaScarrupat dispone di un portale istituzionale raggiungibile all’indirizzo \texttt{www.fantascarrupat.altervista.org}.\\
Il sito costituisce il centro informativo dell’intera Lega: ospita regolamento, calendari, rose, risultati e comunicati ufficiali, custodisce l’albo d’oro e l’archivio storico delle stagioni concluse e raccoglie contenuti multimediali e statistiche approvate dal Consiglio di Lega.

\section{Gestione e responsabilità}
\label{art:10.2}

La gestione tecnica del portale spetta all’Admin di Lega.\\
La correttezza dei contenuti è invece responsabilità solidale di tutti i membri: ciascun consigliere, qualora rilevi inesattezze o lacune, è tenuto a segnalarle tempestivamente all’Admin e a proporre eventuali integrazioni.

\section{Nomi e loghi delle Fantasquadre}
\label{art:10.3}

Ogni nuovo Fantallenatore sceglie un nome e realizza un logo originali per la propria squadra; marchi, stemmi o elementi grafici già esistenti non possono essere riutilizzati, in ossequio alla normativa sul diritto d’autore.\\
Nome e logo vengono pubblicati sul portale e restano vincolanti finché non sia stata autorizzata una modifica secondo l’articolo 10.4.

\section{Procedura di modifica}
\label{art:10.4}

Il Fantallenatore che intenda cambiare nome o logo compila l’apposito modulo “Richiesta cambi nome e logo” e lo inoltra nella chat ufficiale di Lega; l’Admin verifica la conformità ai requisiti e apre una votazione del Consiglio, che delibera con maggioranza assoluta (50 per cento più uno).\\
Le modifiche per motivi urgenti possono essere presentate in qualunque momento; quelle esclusivamente estetiche sono ammesse solo all’inizio della stagione e, se accolte, non potranno essere richieste di nuovo prima che siano trascorse due stagioni complete.\\
In caso di approvazione, l’Admin aggiorna il sito e il materiale ufficiale entro settantadue ore.

\section{Sanzioni}
\label{art:10.5}

Chi modifica nome o logo senza autorizzazione è obbligato a ripristinare immediatamente l’identità approvata.\\
In caso di recidiva si applica un aumento di cinque euro sulla quota di partecipazione, fatto salvo il potere del Consiglio di adottare pene più severe qualora la violazione persista.

\section{Tutela dei dati}
\label{art:10.6}

Il portale opera in conformità al Regolamento (UE) 2016/679 (GDPR). I dati personali dei partecipanti sono trattati esclusivamente per le finalità sportive e sociali della Lega e non vengono ceduti a terzi senza consenso esplicito.



\chapter{Firme e accettazione del regolamento}
\label{cap:firme}

Il presente regolamento costituisce la fonte primaria di norme per la Lega FantaScarrupat. Ogni decisione, sia tecnica sia disciplinare, deve conformarsi alle sue disposizioni; in caso di dubbio interpretativo, il testo qui approvato prevale su qualunque prassi o consuetudine. \\

\noindent
Con la sottoscrizione che segue i partecipanti dichiarano di:
\begin{itemize}
  \item aver letto e compreso integralmente il documento;
  \item approvarlo senza riserve ed impegnarsi a rispettarlo.
\end{itemize}

\begin{longtable}{|c|p{4.5cm}|p{4.5cm}|p{2.8cm}|}
  \hline
  \textbf{N.} & \textbf{Nome e cognome} & \textbf{Firma} & \textbf{Data} \\
  \hline
  1 & Antonio Di Giorgio (P) & & \\
  \hline
  2 & Mario Castaldi & & \\
  \hline
  3 & Cristian Cecere & & \\
  \hline
  4 & Vincenzo Gervasio & & \\
  \hline
  5 & Andrea Lucariello & & \\
  \hline
  6 & Francesco Parolisi & & \\
  \hline
  7 & Lorenzo Parolisi & & \\
  \hline
  8 & Pasquale Lupoli & & \\
  \hline
  \caption{Firme}
\end{longtable}
